\documentclass{article} % Sets the document class to 'article' (good default for most things)

\usepackage[margin=2.5cm]{geometry} % Sets 2.5 cm margin on all sides
\usepackage[utf8]{inputenc} % Allows input of UTF-8 characters (e.g., ä, ö, ü)
\usepackage{amsmath, amssymb} % Math symbols and environments
\usepackage{graphicx} % Allows including images
\usepackage{hyperref} % Makes URLs and references clickable
\usepackage{enumitem} % Customizable lists
\setlength{\parindent}{0pt} % Disable all paragraph indents

\title{My First LaTeX Document} % Title of the document
\author{Your Name} % Author name
\date{\today} % Sets date to today's date

\begin{document} % Start of the content

\maketitle % Generates the title using \title, \author, \date

\section{Introduction} % Creates a numbered section
This is a simple introduction to \LaTeX. % Backslash LaTeX prints the LaTeX logo

\subsection{Text Formatting} % Subsection under Introduction
This is \textbf{bold}, \textit{italic}, and \underline{underlined} text. % Basic text formatting

You can also use \verb|inline code| or \texttt{monospace text}. % Inline code examples

\section{Math Mode} % New section
Inline math: \( a^2 + b^2 = c^2 \) % Inline math in parentheses

Display math:
\[ % Begin display (centered) math environment
\int_0^1 x^2\,dx = \frac{1}{3} % Definite integral from 0 to 1 of x^2, equals 1/3
\] % End display math

\begin{align} % Align equations at equal signs
  a &= b + c \\ % First equation with alignment on '='
  x &= y + z % Second equation with alignment on '='
\end{align} % End of aligned equations block

\section{Lists} % New section for lists
\begin{itemize} % Bullet point list
  \item First item % First bullet point
  \item Second item % Second bullet point
\end{itemize} % End of bullet point list

\begin{enumerate} % Numbered list
  \item First item % First numbered item
  \item Second item % Second numbered item
\end{enumerate} % End of numbered list

\section{Links and References} % New section for links and referencing
Visit \href{https://www.latex-project.org}{LaTeX Project}. % Insert a clickable URL
See Figure~\ref{fig:example}. % Reference a figure by label

\section{Tables} % New section for tables
\begin{tabular}{|c|c|} % Begin tabular environment with two centered columns and vertical bars
  \hline % Horizontal line at the top of the table
  A & B \\ % First row: A and B, separated by '&', end row with '\\'
  \hline % Horizontal line below the first row
  1 & 2 \\ % Second row: 1 and 2
  3 & 4 \\ % Third row: 3 and 4
  \hline % Horizontal line at the bottom of the table
\end{tabular} % End tabular environment

\section{Inserting Images} % Starts a new section titled "Inserting Images"
\begin{figure}[h] % Begins a figure environment; [h] = "here" – place figure approximately here in the text
  \centering % Centers the image horizontally within the figure
  \includegraphics[width=0.5\textwidth]{example-image} % Inserts the image file named "example-image" at 50% of text width
  \caption{An example image} % Adds a caption below the image
  \label{fig:example} % Labels the figure so you can refer to it later with \ref{}
\end{figure} % Ends the figure environment

\section{Conclusion} % New section titled "Conclusion"
This document covered basic LaTeX features for beginners. % Summary of content

\end{document} % End of the content
